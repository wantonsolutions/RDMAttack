\section{Introduction}
\label{sec:intro}

Our attacker model is one in which an attacker has gained access to the victim's
VLAN, and can observe and send traffic. Our goal is to gain both read and write access to a victim computer.

We should note that Throwhammer~\cite{216055} must be run by an adversary, if
the adversary has the ability to spoof requests, then we cannot defend against
it, but if we assume that Throwhammer requests only come from malicious
clients, then we can say that our defense protects against throwhammer as well.
We should also state that our attaker model, if we are able to modifiy headers,
could instantiate a throwhammer attack by modifing the memory addresses of the
requests.

We should also determine if an attacker can perform a denial of service by
modifing RDMA packets. The idea here is that the high speed nature of RDMA
makes it an ideal DOS tool.

We should also state our potential goals. For example if all we want is
authenticity we only need to negotiate keys for the RDMA session. However, if
we want, confidentiality we may be hosed. -- After a conversaiton about
confidentiality it apears that it is on a spectrum, and that the degree of
confidentiality something has is based on the guessing abilities of an
adversary.

