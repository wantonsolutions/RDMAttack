\section{Abstract}
\label{sec:abstract}

RDMA over converged ethernet (RoCEv2) is a high performance network protocol
for directly accessing the memory of a remote machine by bypassing its CPU.
The protocol's design prioritizes performance above all else, leaving security
properties such as authenticity and confidentiality by the wayside. Security is
largely unnecessary in the intimately coupled compute clusters RoCEv2 was
designed for. RDMA's performance benifits have been widely recognized, and it
has been integrated into a wide set of dataceter applications. Many of these
general purpose applications run on diverse sets of machines (sometimes across
data centers) which has propelled RDMA into high risk environments it was never
designed for.

In this work we consider the design of secure RDMA. As motivation for the redesign,
we hijack RDMA by performing a trivial man in the middle attack. We show that
an unsophisticated attacker in control of a switch's routing table can gain
full control over plaintext RDMA payloads, reading and writing to arbitrary victim addresses.
Securing RDMA requires careful consideration of its performance; practitioners
will not adopt a solution which degrades their high performance applications.
We benchmark common encryption algorithms and demonstrate that commodity CPU's
cannot encrypt at the beefy bandwidth's (100 and 400Gbps) RDMA is designed for.
Our conclusion is that RDMA encryption must be implemented as a NIC offloaded
function to achieve line rate performance.
