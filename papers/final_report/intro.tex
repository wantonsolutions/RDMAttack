\section{Introduction}
\label{sec:intro}

In ultra high performance distributed applications servicing requests for data
with a CPU is a faux pas. The latency incurred by having a CPU
service a network interrupt, copy memory to a buffer then to the Network
Interface Card (NIC), and trigger a send is considered unseemly. To save face
protocols for bypassing the CPU which allow Remote Direct Memory Access (RDMA)
by NIC's are supported by most well-to-do operating systems. RDMA over
converged ethernet is an en vogue instantiation of RDMA among elite data centre
practitioners. In the high performance upper crust of data centre clusters
security is an after thought as their traffic is largely partitioned from
riff-raff applications run by potentially malicious individuals. As such RoCEv2
was designed as a separate class of protocol with minimal security considerations.

RoCE packets are bare unencrypted bytes, easily voyeurable by an attacker. The
scant protection provided by the protocol comes as a checksum (iCRC) and 16 bit
partition identification key used by the IOMMU to reject accesses to memory
addresses not registered for RDMA. These minimal mechanisms for integrity and
protection leave RoCE exposed to attackers. A man in the the middle can peek
into every read revealing potentially sensitive program state, or simply modify
the packet entirely reading or writing to arbitrary positions in a victims
address space.

Securing RoCE, due to its performance aims, requires care. Network line rate
rages towards terrabits per seconds, and RDMA is expected to scale in order to
saturate links. Any encryption used to secure the protocol must operate at a
competitive rate or risk becoming a latency bottleneck. Additionally part's of
RoCE protocol are offloaded to the NIC itself, meaning that any encryption
performed by the CPU must be carefully coordinated between the two
asynchronous hardware devices.

In this work we Attack RDMA; first we show how a trivial man in the middle
attack can be performed on the protocol by any attacker able to own a switch
between two RDMA enabled hosts. We demonstrate the power of the attack by
demonstrating full control over the protocol by an attacker: the ability to
observer all traffic, modify payloads allowing reads and writes to arbitrary
locations within an RDMA enabled region. We also demonstrate that an attacker
can steer RDMA traffic to perform throwhammer attacks. Post attack we discuss
the design space of a solution for secure RDMA, we contrast the bandwidths
attainable by a variety of CPU encryption algorithms and NIC offloaded
encryption. Our finding lead us to believe that any standard secure RDMA
algorithm must be implemented as an offloaded function to the NIC itself, in
order not to interfere with 400, and potential terabit network bandwidths. 

In Section (Background) we supply background to the RDMA protocol


